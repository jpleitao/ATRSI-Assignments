\documentclass[11pt]{article}
\usepackage[english]{babel}
\usepackage{natbib}
\usepackage{url}
\usepackage[utf8]{inputenc}
\usepackage{amsmath}
\usepackage{amssymb}
\usepackage{graphicx}
\usepackage{parskip}
\usepackage{fancyhdr}
\usepackage{vmargin}
\usepackage{booktabs}
\usepackage[table,xcdraw]{xcolor}
\usepackage{tabularx}
\usepackage{caption} 
\usepackage{float}
\usepackage{longtable}
\usepackage{array}
\usepackage{caption}
\usepackage{subcaption}

\setmarginsrb{3 cm}{1 cm}{3 cm}{1 cm}{1 cm}{1.5 cm}{1 cm}{1.5 cm}

\newcolumntype{L}[1]{>{\raggedright\let\newline\\\arraybackslash\hspace{0pt}}m{#1}}
\newcolumntype{C}[1]{>{\centering\let\newline\\\arraybackslash\hspace{0pt}}m{#1}}
\newcolumntype{R}[1]{>{\raggedleft\let\newline\\\arraybackslash\hspace{0pt}}m{#1}}

\title{Assignment \#4 - Process Modelling with Neuro-Fuzzy Systems}
\date{\today}

\makeatletter
\let\thetitle\@title
\let\thesubtitle\@subtitle
\let\theauthor\@author
\let\thedate\@date
\makeatother

\pagestyle{plain}

\captionsetup[table]{skip=5pt}


\begin{document}

%%%%%%%%%%%%%%%%%%%%%%%%%%%%%%%%%%%%%%%%%%%%%%%%%%%%%%%%%%%%%%%%%%%%%%%%%%%%%%%%%%%%%%%%%

\begin{titlepage}
	\centering
    \textsc{\LARGE University of Coimbra}\\[1.0 cm]
	\textsc{\large Doctoral Program in Information Science and Technology}\\[0.5 cm]
    \textsc{\large Real Time Learning in Intelligent Systems}\\[5 cm]
	\rule{\linewidth}{0.2 mm} \\[0.4 cm]
	{ \LARGE \bfseries \thetitle}\\ [0.2 cm]
    \rule{\linewidth}{0.2 mm} \\[3 cm]
    
    \textsc{Joaquim Pedro Bento Gonçalves Pratas Leitão - 2011150072}\\[5 cm]
	
	{\large \thedate}\\[2 cm]
 
	\vfill
	
\end{titlepage}

%%%%%%%%%%%%%%%%%%%%%%%%%%%%%%%%%%%%%%%%%%%%%%%%%%%%%%%%%%%%%%%%%%%%%%%%%%%%%%%%%%%%%%%%%

\section{Introduction}
\label{introduction}

Dynamic systems are often described and modelled by different equations, where the outputs in a given time instant (say k) are conditioned by previous outputs (in instants $k-1$, $k-2$, ...) and inputs (in instants $k$, $k-1$, $k-2$, ...). A system is also said to have \emph{inertia} when the inputs in a given instant only affect its outputs in posterior instants (that is, input at instant $k$ can only influence the output at instants $k+1$, $k+2$, ...).

The current assignment proposes the development of a Sugeno-type Neuro-Fuzzy System (NFS) to model the dynamics of a given process or system with inertia. NFSs of this nature are characterised by mapping their input space to an output space using a series of fuzzy \emph{if-then} rules. In the particular case of Sugeno NFSs the output of each rule is written as a linear combination of the input variables. In simpler Sugeno systems, this linear combination consists in a constant value.

By collecting data containing pairs of the system's input and corresponding output values the rules that define the NFS can be learned in such a way that they describe the system's behaviour.

In other words, exploring collected data from the system the mentioned fuzzy rules can be learned, resulting in the development of a Sugeno NFS that models the dynamics of the desired process or system, as it is the objective of this assignment.

The remainder of this document is organised as follows: Section \ref{modelled_system} presents the system to be modelled in this work; Section \ref{methodology} describes the methodology followed throughout the project; Sections \ref{fase_a} and \ref{fase_b} cover the main steps of the work, presented in the Methodology section. Finally, Section \ref{conclusion} concludes the document.


\section{Modelled System}
\label{modelled_system}

The system to be modelled in this work can be defined by the following transfer function:

$$ G(s) = \frac{2}{s^{3} + 5s^{2} + 6.75s + 2.25}$$

Since the provided transfer function features $3$ poles and no zeros - therefore it has more poles than zeros - the presented system is a third-order system. It can be proved that the provided system has memory of $3$ instants in both the inputs and outputs and \emph{inertia} (meaning that inputs at a given time instant only influence future outputs). In this sense, the output of the system at any time instant $k$ can be written as:

$$ y(k) = f(y(k-1), y(k-2), y(k-3), u(k-1), u(k-2),u(k-3))$$

where $y(k)$ represents the system's output at instant $k$ and $u(k)$ represents the system's input at instant $k$.

\section{Methodology}
\label{methodology}

The project and development of a Neuro-Fuzzy System can be a complex and extensive task. Nevertheless, in the scope of the current work, the following two main steps will be considered:

\begin{enumerate}
	\item \textsc{Learning Stage}, where a model of the system is learned using data collected from the system. This stage comprises three main sub-tasks:
	
	\begin{itemize}
		\item \emph{Data collection}, where pairs of system's \emph{input-output} data are collected, which will be used as a labelled dataset during the learning. Collecting a good set of \emph{input-output} data is critical in the final outcome. These data should be representative of the dynamics of the system, making the input and output vary over the entire permissible domain. A good technique is to use a random input sequence that forces the output to move through the entire space.
		
		\item \emph{Fuzzy rules initialisation}. An initial estimate for the NFS's rules is obtained by applying a \emph{clustering} algorithm to the labelled dataset collected in the previous point. Common algorithms used at this stage include subtractive, c-means and fuzzy c-means clustering.
		
		\item \emph{Fuzzy rules optimisation}. A further optimisation of the fuzzy rules computed in the previous point is usually performed, aiming to reduce the modelling error. An \emph{ANFIS} structure is commonly used along with either a \emph{backpropagation} or \emph{hybrid} optimisation method (although other alternatives maybe considered at this point).
		
	\end{itemize}
	
	\item \textsc{Assessment Stage}, where different input signals are provided to both the developed model and the real system with the objective of comparing the real and simulated outputs. The ultimate goal of this task is to certify that the developed model of the system can accurately describe its behaviour.
\end{enumerate}

\section{Learning Stage}
\label{fase_a}

Detalhar o que se fez.

Começar pela geração do dataset, referir clustering methods e optimisation methods.

Não esquecer das funções de pertença em cada caso!!

\section{Assessment Stage}
\label{fase_b}

Melhorar título e detalhar a segunda fase


\section{Conclusion}
\label{conclusion}

\end{document}