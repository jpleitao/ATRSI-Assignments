\documentclass[11pt]{article}
\usepackage[english]{babel}
\usepackage{natbib}
\usepackage{url}
\usepackage[utf8]{inputenc}
\usepackage{amsmath}
\usepackage{amssymb}
\usepackage{graphicx}
\usepackage{parskip}
\usepackage{fancyhdr}
\usepackage{vmargin}
\usepackage{booktabs}
\usepackage[table,xcdraw]{xcolor}
\usepackage{tabularx}
\usepackage{caption} 
\usepackage{float}
\usepackage{longtable}
\usepackage{array}
\usepackage{caption}
\usepackage{subcaption}

\setmarginsrb{3 cm}{1 cm}{3 cm}{1 cm}{1 cm}{1.5 cm}{1 cm}{1.5 cm}

\newcolumntype{L}[1]{>{\raggedright\let\newline\\\arraybackslash\hspace{0pt}}m{#1}}
\newcolumntype{C}[1]{>{\centering\let\newline\\\arraybackslash\hspace{0pt}}m{#1}}
\newcolumntype{R}[1]{>{\raggedleft\let\newline\\\arraybackslash\hspace{0pt}}m{#1}}

\title{Assignment \#4 - Process Modelling with Neuro-Fuzzy Systems}
\date{\today}

\makeatletter
\let\thetitle\@title
\let\thesubtitle\@subtitle
\let\theauthor\@author
\let\thedate\@date
\makeatother

\pagestyle{plain}

\captionsetup[table]{skip=5pt}


\begin{document}

%%%%%%%%%%%%%%%%%%%%%%%%%%%%%%%%%%%%%%%%%%%%%%%%%%%%%%%%%%%%%%%%%%%%%%%%%%%%%%%%%%%%%%%%%

\begin{titlepage}
	\centering
    \textsc{\LARGE University of Coimbra}\\[1.0 cm]
	\textsc{\large Doctoral Program in Information Science and Technology}\\[0.5 cm]
    \textsc{\large Real Time Learning in Intelligent Systems}\\[5 cm]
	\rule{\linewidth}{0.2 mm} \\[0.4 cm]
	{ \LARGE \bfseries \thetitle}\\ [0.2 cm]
    \rule{\linewidth}{0.2 mm} \\[3 cm]
    
    \textsc{Joaquim Pedro Bento Gonçalves Pratas Leitão - 2011150072}\\[5 cm]
	
	{\large \thedate}\\[2 cm]
 
	\vfill
	
\end{titlepage}

%%%%%%%%%%%%%%%%%%%%%%%%%%%%%%%%%%%%%%%%%%%%%%%%%%%%%%%%%%%%%%%%%%%%%%%%%%%%%%%%%%%%%%%%%

\section{Introduction}
\label{introduction}

Dynamic systems are often described and modelled by different equations, where the outputs in a given time instant (say k) are conditioned by previous outputs (in instants $k-1$, $k-2$, ...) and inputs (in instants $k$, $k-1$, $k-2$, ...). A system is also said to have \emph{inertia} when the inputs in a given instant only affect its outputs in posterior instants (that is, input at instant $k$ can only influence the output at instants $k+1$, $k+2$, ...).

The current assignment proposes the development of a Sugeno-type Neuro-Fuzzy System (NFS) to model the dynamics of a given process or system with inertia. NFSs of this nature are characterised by mapping their input space to an output space using a series of fuzzy \emph{if-then} rules. In the particular case of Sugeno NFSs the output of each rule is written as a linear combination of the input variables. In simpler Sugeno systems, this linear combination consists in a constant value.

By collecting data containing pairs of the system's input and corresponding output values the rules that define the NFS can be learned in such a way that they describe the system's behaviour.

In other words, exploring collected data from the system the mentioned fuzzy rules can be learned, resulting in the development of a Sugeno NFS that models the dynamics of the desired process or system, as it is the objective of this assignment.

The remainder of this document is organised as follows: Section \ref{modelled_system} presents the system to be modelled in this work; Section \ref{methodology} describes the methodology followed throughout the project; Sections \ref{fase_a} and \ref{fase_b} cover the main steps of the work, presented in the Methodology section. Finally, Section \ref{conclusion} concludes the document.


\section{Modelled System}
\label{modelled_system}

Dizer que o sistema é de 3ª ordem (3 pólos)e uma vez que tem mais pólos (3) do que zeros (0) é um sistema com inércia.

O sistema a ser considerado neste trabalho é descrito pela seguinte função de transferência:

$$ G(s) = \frac{2}{s^{3} + 5s^{2} + 6.75s + 2.25}$$

Pode-se demonstrar que este sistema tem memória na saída e na entrada até ao instante 3, ou seja:

$$ y(k) = f(y(k-1), y(k-2), y(k-3), u(k-1), u(k-2),u(k-3))$$

onde $y(k)$ representa a saída do sistema no instante $k$ e $u(k)$ a entrada do sistema no instante $k$.

\section{Methodology}
\label{methodology}

Apresentar resumidamente as diferentes etapas que compõem este trabalho: as duas fases e detalhar cada uma delas.

\section{Fase A}
\label{fase_a}

Melhorar título e detalhar o que se fez

Não esquecer das funções de pertença!!

\section{Fase B}
\label{fase_b}

Melhorar título e detalhar a segunda fase


\section{Conclusion}
\label{conclusion}

\end{document}